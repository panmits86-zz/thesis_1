% Chapter 6

\chapter{Concluding Remarks and Discussion} % Main chapter title

\label{Chapter6} % For referencing the chapter elsewhere, use \ref{Chapter6}


Observations are essential to define the wind and wave conditions through all phases of an offshore wind farm, from the design and planning to decommission. They also enhance the quality of the numerical models' initial conditions estimation. In this study, data from buoys and satellite altimeters in the SNE region were used and analyzed.

First, we specified the normal conditions in different scales of variability using the buoy measurements due to their continuous, long-term, and reliable datasets. During the summer season, the surface WS diurnal range can reach up to 2m/s in stations with a distance of 40 kilometers or less from the closest coast. These values are lower than the diurnal range on a meteorological station on land, and the uncertainty is generally higher. The WS diurnal variability also has unique characteristics over the ocean. WS minimum values are identified six hours after the lowest WS on land and the maximum values three to six hours later. The wind direction diurnal cycle has similar patterns for the on land and offshore stations, with a lower diurnal range during the summer season for the coastal buoys. All the above do not apply for stations located over 100 kilometers off the coast, where the diurnal variability is insignificant.

SNE, a mid-latitude North Atlantic coastal region, is characterized by substantial WS and SWH seasonal variability. On the one hand, it ranges from 2m/s for the locations close to the coast to almost 4.5 m/s for the open ocean buoys. The SWH seasonal variability is negligible in areas with proximity to land and a low presence of swell waves. In contrast, it increases with distance to the coast, and it reaches up to 1.5 meters for the open ocean buoys. This difference is only partly explained by the direct influence of the wind on the sea surface. Wind roses sufficiently visualize the directional wind distribution. Generally, the wind has a west or northwest direction during the winter, while it comes from the west or southwest during the summer months. Directional wave distribution does not show the same homogeneity, especially during the winter months when higher wind speeds, storms, and swells with higher energy density are present. The directional wave spectrum of buoy 44097, located on the eastern side of SNE, reveals that the highest monthly average energy density is attributed to waves that arrive from remote regions in the Atlantic Ocean. We can also identify three systems developing throughout the year. The southeastern swell, which has energy density peaks in November and March, is the most dominant. The results indicate the importance of the directional wave spectrum for studying and monitoring the wave conditions in the SNE. Directional spectra observations can also be derived from SAR and satellite missions focused on wind and waves (CFOSAT) and assimilated into numerical models to correct their initial conditions. Therefore, the reconstruction of the directional spectrum of waves from wave buoys or NDBC stations, when available, can serve as a ground truth reference for future studies involving remote sensing observations and the evaluation of numerical models.

The relatively small number of available years of data for most buoys and the gaps in their time-series record were limiting factors for detecting statistically significant WS and SWH trends in SNE. Studies on general or extreme WS and SWH trends using in situ, remote sensing observations, and data from models show inconsistent or inconclusive results \cite{Vose2014} for the Eastern US coasts. On the other hand, the increased storm frequency and intensity in coastal regions like SNE indicate the need for further investigation.

Extreme events are also critical for offshore wind energy yield estimation and fatigue assessment on the infrastructure. Therefore, the wind and wave conditions before, during, and after the passing of a nor’easter storm are examined as a case study.

The relationships between $u_{10}$ and $H_{s}$ are calculated with polynomial regression fitting to the buoy observations for each of the primary wind directions. The relationships can adequately represent the wind and wave conditions in areas where the wind influence prevails, like the Nantucket Sound, but they are insufficient in swell-dominated locations. For this reason, the wind-wave coupling was further examined using the inverse wave age criterion that considers the wave celerity with respect to $u_{10}$. This process results in the classification of waves based on their growth, and its advantage is the inclusion of the wind and mean wave direction. The classification shows that at the coastal buoys’ locations, especially on the western side of the SNE region, the wave conditions are influenced by the wind, notably during the winter. Still, there is also a substantial presence of swell waves, which is increased during the summer months due to the decreasing wind speed. 70\% of the observations at the sheltered buoy 44020 location consist of purely wind waves, while at the open ocean buoy 44008, the sea state is swell-dominated with 95\% mixed and swell waves during the summer. We need to emphasize that relationships and classifications have their limitations, primarily because they consider only the wave spectrum's peak. A complete way to observe the sea state is by considering the whole directional wave spectrum, if available.

The accurate estimation of the WS PDF and its parameters is crucial for the offshore wind energy assessment. Specifically, the shape and scale parameters of the distribution are included in calculating the wind power output. The Weibull distribution is the most commonly used and proposed as the best fit to the WS data in similar studies. After fitting the data from four buoys located less than 40 kilometers offshore to a collection of 90 distributions and evaluating the results visually with the probability plots and quantitatively with error statistics, we suggest that the Johnson $S_{B}$ and the Beta distributions generally have exceptional performance. Only data from the sheltered buoy 44020 fit best to the Weibull 3P. The limitation of such estimations for offshore wind is that the rotor's height is greater than or equal to 100 meters, which is substantially higher than the 10-meter buoy reference height. Future studies using data from offshore towers and lidar buoys in the domain that measure WS and direction at several heights will also expand our knowledge by comparing the results included herein. Besides, even if the distributions mentioned above fit exceptionally the data from buoys in SNE, we need to examine whether mixed distribution models proposed in similar studies \cite{Morgan2011} improve our estimations.

Satellite altimetry is also an essential source of offshore wind data due to SWH and WS observations' availability to characterize both the sea state and the surface wind regime. Validation of altimeter WS and SWH against in situ observations is required to assure the quality and determine the measurements' accuracy. Although SNE is a relatively small domain with a limited number of available stations, the agreement between the collocated altimeter and buoys dataset is confirmed. The consistency is highlighted by correlation coefficient maps showing high values (over 0.9) at each buoy location to compare using the collective altimeter dataset and an increasing trend with increasing distance from the land. The results verify the exceptional performance of SARAL-AltiKa close to the coast and its ability to provide reliable measurements in low sea states.

The accumulated dataset from all five satellite altimeters was used to interpolate the observations using the kriging methodology for the 2019 winter and summer seasons. The resulting maps show strong WS gradients during winter in SNE, and further investigation is needed to assess the sensitivity of the output maps to extreme events. SWH maps show coherency during winter and summer with the highest sea states at the southeastern part of the region and decreasing SWH with increasing proximity to the land. Although the availability of SWH and WS in locations close to the coast or areas surrounded by land is one of the advantages, the estimation is not as robust as in the open ocean due to the smaller sample size and the limitations induced by topography. The various sources of errors in the estimation are documented, emphasizing the WS diurnal or sampling bias. Monitoring of the sea state and surface wind conditions requires multiple sources of data, and combined satellite datasets are analyzed in recent studies \cite{Hasager2015}. The requirement to fill the temporal gaps of the satellite altimeter observations in future studies is addressed.